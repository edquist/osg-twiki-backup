\documentclass[12pt]{article}


\usepackage{fullpage}

% - use \affiliation{author1, author2, ... and authorN}{address} for author(s)  <typeset in italics>
%         with a single affiliation (address)\def\affiliation#1#2{\gdef\@address{}

\begin{document}
\title{GLUE Schema v1.2 Mapping to Old ClassAd Format\\
July 24, 2006}
\author{
Sergio Andreozzi (\em{INFN})\\
Gabriele Garzoglio, Sudhamsh Reddy (\em{Fermilab})\\
Marco Mambelli (\em{University of Chicago})\\
Alain Roy (\em{University of Wisonsin}),\\
Shaowen Wang (\em{University of Iowa}),\\
Torre Wenaus (\em{BNL})}

\maketitle

\begin{abstract}
\noindent This document describes a mapping between the GLUE schema
v1.2 and old ClassAd format. The old ClassAd representation of
resource characteristics can be used in out-of-the-box resource
selection technologies, such as the Condor Matchmaking Service. This
mapping has been written focussing on the problem of resource
selection and is presented as a series of rules with examples.
\end{abstract}

\newpage

\tableofcontents

\newpage

\section{Introduction}
\label{se:Introduction}

The GLUE schema \cite{GLUE-schema} provides a model to describe Grid
resources and their relationship. Previous works \cite{LDIF-mapping}
\cite{XML-mapping} present a mapping between the GLUE schema and
different popular data formats, such as LDIF and XML. This document
presents a mapping between the GLUE schema and the old ClassAd
format. Such representation can be immediately used by
out-of-the-box software, such as the Condor Matchmaking Service, to
implement resource selection.

\section{Limitations}
\label{se:limitation}

In this document, the main motivation for the mapping of the GLUE
schema to old ClassAd format is providing a resource description
that can be easily used in resource selection and discovery
services. We do not focus on other potential uses of such mappings,
such as resource monitoring or job environment enhancing. With ``job
environment enhancing'' we intend the process of adding to the job
environment by dereferencing abstract attribute values to concrete
values, after resource selection is completed \footnote{In condor
ClassAd, attribute dereferencing is expressed with the
\$\$(AttributeName) syntax}. For these reasons, some entities of the
GLUE schema, such as ``Job'' of ``Computing Element'', or
the ``Path'' attribute of the ``Location'' entity, are explicitly
excluded from the mapping.

In addition, this document does not provide a mapping for the
Storage Resources description of the GLUE schema. This mapping
addresses the translation to old ClassAd format for the computing
resources only.

\section{Mapping Rules}
\label{se:MappingRules} This section presents a set of rules to
translate the GLUE schema into old ClassAd format. The section
describes a set of basic rules and a set of extended rules. These
rules apply to the GLUE schema v1.1 and v1.2.

\subsection{Implementations}
\label{se:Implementations} As of March 2006, the basic rules have
been implemented as the ``old\_ClassAd'' dialect of the CEMon
service (v1.7.2) from gLite \cite{CEMon}.

The mapping uses as input the GLUE representation of the resources
in LDIF format \cite{LDIF-mapping,LDIF} from the Generic Information
Providers \cite{GIP}. As such, it should be noted that it is not
mandatory for all the GLUE attributes to have defined values at all
sites. Developers should keep this in mind when implementing code
that works with this mapping e.g. Condor matchmaking functions.

The extended rules are an extension/modification to the basic rules and
focus on the ``VOView'' and ``Location'' entities in the GLUE schema v1.2.
This extended mapping has only a partial implementation in CEMon v1.7.2
as of May 2006.


\subsection{Basic Rules}
\label{se:BasicRules}

In this section, we describe the basic rules of the mapping of LDIF
representation of Grid resources based on the GLUE Schema into the
old ClassAd format.

\subsubsection{Definitions}

\begin{itemize}
\item Attribute names are built using the same rules as for the LDIF
mapping \cite{LDIF-mapping} and are not described here.

\item The mapping is built by considering all the possible combinations of
inter-related ComputingElement, Cluster, and Subcluster entities. In
other words, each combination contains a single ComputingElement,
Cluster, and Subcluster entity. Attributes in each combination are
then mapped to a single old ClassAd. In general, the full
representation of the GLUE schema for a site in old ClassAd format
consists of a set of old ClassAds. Each ClassAd represents a sort of
virtual homogeneous cluster with a single entry point and waiting
queue.

\item Attributes of the CE, Cluster, and Subcluster entities that,
according to the schema, do not appear with more than one value
(e.g. GlueCEInfoHostName, GlueCEInfoTotalCPUs) are translated in old
ClassAd format as AttributeName = Value

\item Attributes that can appear with more than one value (indicated in the UML
representation of the GLUE schema with an asterisk [*] next to the
attribute type or without the keyword \verb+SINGLE-VALUE+ in the
LDAP attribute definition), must be treated differently. In an old
ClassAd, in fact, attributes must be unique, i.e. they appear only
once in the ClassAd. Every GLUE attribute that can be present
multiple times with different values (e.g.
GlueHostApplicationSoftwareRunTimeEnvironment) appears only once in
the old ClassAd mapping. The multiple values are represented as a
single comma-separated string of the original values. For the
purpose of resource selection using the Condor Matchmaking service,
the values can be individually used in expressions using custom
(call-out) functions. For example, to select a ClassAd that contains
a certain value in one of these attributes, the Matchmaking Service
could provide the users a syntax like
\begin{verbatim}
  MultiValueAttribute ==
       matchString("myValue",MultiValueAttributeValues)
\end{verbatim}


\item In some General Information Provider implementations (GIP),
the ContactString Info attribute of the ComputingElement entity is
not necessarily a GRAM URL and may have the form
$<$GRAM-URL$>$-$<$VO-Name$>$. This form cannot be used directly as a
contact string by most Grid clients. In LCG, the Resource Broker
processes this value so that it can be used by the standard Grid
clients. In the mapping to old ClassAd, instead, the value is
published in the form $<$GRAM-URL$>$, irrespectively of the GIP
implementation, by comparing the value of ContactString with the
values in the GlueCEAccessControlBaseRule (i.e., the VO names). This
is important in order to use the Condor Matchmaking Service
out-of-the-box for Resource Selection.


\item The ``Job'' and ``VOView'' entities of ``Computing Element'' AND the
``Location" entity of "SubCluster" are not considered for the
mapping to old ClassAd in the basic mapping rules. ``VOView'' and
``Location'' are used in the extended mapping rules.

\item The following rules are followed to quote string values in old ClassAd
\begin{verbatim}
ClassAd Keywords
  Special keyword are not quoted.
  Special keywords: undefined, true, t, false, f, error
  Note: int, float, and the like are not keywords in the
  ClassAd language and have no special meaning

Variables
  If an attribute refers to another attribute, the attribute
  name is a variable, and is not quoted.
  For instance, if you have:
  X = 3
  Y = X
  then if you evaluate Y, you get 3. In this case, no quote
  marks are used.
  Variables can be combined into expressions, like:
  Y = X + 3
  Y = X > 3

Numbers
  Numbers are not quoted, unless you want them to be
  interpreted as strings. There are floating point numbers
  and integers. If the C function strtod() recognizes it as
  a floating point number, then it is, otherwise it isn't.

Strings
  The only other element that you can have is a string,
  and it must be quoted.
  X = "blah"
  Y = X
  Z = "foo"
  A backslash (\) can also be used for quoting the following
  strings:
  \  (\<blank>)
  \"
  \n
  \r
\end{verbatim}
\end{itemize}

\subsubsection{Considerations}

These rules are useful when using the Condor Matchmaking Service to
implement Resource Selection. These rules are
implementation-specific, thus are kept outside of the ``generic''
mapping rules.

\begin{itemize}
\item When using the Condor Matchmaking Service for Resource Selection,
every generated ClassAd should have the following three attributes
added to it: ``MyType'', ``TargetType'', and ``Name''. These
attribute are used internally by Condor for Matchmaking and have the
following constant values:

MyType = "Machine"

TargetType = "Job"

Name = $<$GlueCeName$>$.$<$GlueClusterName$>$.$<$GlueSubclusterName$>$

where the notation $<$GlueCeName$>$ indicates the value of the GlueCeName
attribute and similarly for the other attributes.

It should be noted that the attribute ``Name'' must be unique among
all ClassAds considered for resource selection.
\end{itemize}

\subsubsection{Example}
\label{se:BasicRulesExample}

Firgure \ref{CLASSAD-Basic-1-Example} and
\ref{CLASSAD-Basic-2-Example} show an example of a set of two old
ClassAds. The set is derived applying the ``basic rules'' (section
\ref{se:BasicRules}) to the LDIF representation of the example GLUE
schema in figures \ref{LDIF-CE-1-Example} \ref{LDIF-CE-2-Example},
\ref{LDIF-Cluster-Subcluster-Example},
\ref{LDIF-VOView-Site-Example}, and \ref{LDIF-Location-Example}.

Note that the LDIF example is derived from a real configuration of
resources and does not include the GLUE attributes
``GlueClusterTmpDir'', ``GlueClusterWNTmpDir'',
``GlueHostArchitecturePlatformType'', ``GlueHostProcessorVersion'',
``GlueHostProcessorInstructionSet'',
``GlueHostProcessorOtherDescription'', and
``GlueCEInfoGRAMVersion''. These attributes are included for
completeness in the old ClassAd set example.



\subsection{Extended Rules}
\label{se:ExtendedRules}

In this section, we describe the extended rules of the mapping of
LDIF representation of Grid resources based on the GLUE Schema into
the old ClassAd format. The extended rules provide a mapping to old
ClassAd for the GLUE schema entities "VOView" and "Location". The
extended rules have to be used in addition to the basic rules of
section \ref{se:BasicRules}.


\subsubsection{Definitions}

The extended rules are the following:

\begin{itemize}

\item The mapping is built by considering all the possible combinations of
inter-related ComputingElement, Cluster, Subcluster, and VOView
entities. In other words, each combination contains a single
ComputingElement, Cluster, Subcluster, and VOView entity. Attributes
in each combination are then mapped to a single old ClassAd. VOView
attributes that have the same names as the ones for CE overwrite the
CE values (e.g. GlueCEStateFreeJobSlots from the linked VOView, if
present, overwrites GlueCEStateFreeJobSlots from CE). This makes it
easier for users to express requirements on resource characteristics
that affect the VO/group/role. In fact, the requirement expression
does not change depending on whether VOView is present or not.

\item After building all ClassAds, it should be checked whether the value of 
the attribute GlueCEAccessControlBaseRule of the CE entity is the same as
one of the GlueCEAccessControlBaseRule attributes in the 
VOView entities. If this is NOT the case, the CE entity refers to an 
Access Control Base Rule that is not qualified using a VOView entity.
Therefore, in order to advertise this CE-level Access Rule, an extra set
of ClassAds should be generated by building all possible combinations of the
inter-related ComputingElement, Cluster, Subcluster only (excluding 
the VOView entities). If this were not done, the GlueCEAccessControlBaseRule
attribute of the CE would go lost, as the GlueCEAccessControlBaseRule
attributes of VOView always overwrite the one for CE.

\item The location entity appears in old ClassAd as a multivalue attribute.
The mapping is built as follows. For a given subcluster entity, we
consider all its related Location entities. For a given attribute in
the Location entity, we build the list of values in all the related
Location entities as a comma-separated string of values. The order
of the values must be the same across different attributes, i.e.
values from the same entity have the same position on the list. This
way, pairs of values from two different Location attributes can be
selected using custom (call-out) functions in the Condor Matchmaking
Service. Example signatures for such functions could be

\begin{verbatim}
  matchOrderedPair(AttributeName1, AttrbuteName2,
                   Value1, Value2)
\end{verbatim}
    or
\begin{verbatim}
  matchOrderedNTuple(NumberOfAttributes,
                     AttributeName1, AttributeName2, ... ,
                     Value1, Value2, ...)
\end{verbatim}

In summary, in old ClassAd format, for every subcluster, every
attribute in the Location entity appears once with the list of
values expressed as a list.

\item The attribute Path from the Location entity is not advertised, as it is
deemed to be of marginal importance in the resource selection process.

\item At least these two attributes from the Site entity are provided: 
GlueSiteUniqueID and GlueSiteName. In particular, the GlueSiteName is of
interest in the selection of resources. 

\end{itemize}

\subsubsection{Example}
\label{se:ExtendedRulesExample}

Figure \ref{CLASSAD-Extended-1-Example-part1},
\ref{CLASSAD-Extended-1-Example-part2},
\ref{CLASSAD-Extended-2-Example-part1}, and
\ref{CLASSAD-Extended-2-Example-part2}  show an example of a set of
two old ClassAds (each ClassAd is split in two figures to fit on the
page). The set is derived applying the ``extended rules'' (section
\ref{se:ExtendedRules}) to the LDIF representation of the example
GLUE schema in figures \ref{LDIF-CE-1-Example}
\ref{LDIF-CE-2-Example}, \ref{LDIF-Cluster-Subcluster-Example},
\ref{LDIF-VOView-Site-Example}, and \ref{LDIF-Location-Example}.

The ClassAd are identical to the ones generated with the basic
rules, with the addition of Location attributes. VOView attributes
have overwritten CE attributes in the case of the cdf VO. The
GlueLocationPath attribute of the Location entity is not included in
the mapping.


\begin{figure}
\scriptsize
\begin{verbatim}
# Computing Element Entity (1)
dn: GlueCEUniqueID=rsgrid3.its.uiowa.edu:2119/jobmanager-condor-cdf,mds-vo-name=local,o=grid
objectClass: GlueCETop
objectClass: GlueCE
objectClass: GlueSchemaVersion
objectClass: GlueCEAccessControlBase
objectClass: GlueCEInfo
objectClass: GlueCEPolicy
objectClass: GlueCEState
objectClass: GlueInformationService
objectClass: GlueKey
GlueCEHostingCluster: rsgrid3.its.uiowa.edu
GlueCEName: cdf
GlueCEUniqueID: rsgrid3.its.uiowa.edu:2119/jobmanager-condor-cdf
GlueCEInfoGatekeeperPort: 2119
GlueCEInfoHostName: rsgrid3.its.uiowa.edu
GlueCEInfoLRMSType: condor
GlueCEInfoLRMSVersion: 6.7.13
GlueCEInfoTotalCPUs: 1
GlueCEInfoJobManager: condor
GlueCEInfoContactString: rsgrid3.its.uiowa.edu:2119/jobmanager-condor
GlueCEInfoApplicationDir: /export/grid/app
GlueCEInfoDataDir: /export/grid/data
GlueCEInfoDefaultSE: rsgrid3.its.uiowa.edu
GlueCEStateEstimatedResponseTime: 0
GlueCEStateFreeCPUs: 0
GlueCEStateRunningJobs: 0
GlueCEStateStatus: Production
GlueCEStateTotalJobs: 0
GlueCEStateWaitingJobs: 0
GlueCEStateWorstResponseTime: 0
GlueCEStateFreeJobSlots: 1
GlueCEPolicyMaxCPUTime: 0
GlueCEPolicyMaxRunningJobs: 0
GlueCEPolicyMaxTotalJobs: 0
GlueCEPolicyMaxWallClockTime: 0
GlueCEPolicyPriority: 0
GlueCEPolicyAssignedJobSlots: 1
GlueCEAccessControlBaseRule: VO:cdf
GlueForeignKey: GlueClusterUniqueID=rsgrid3.its.uiowa.edu
GlueInformationServiceURL: ldap://rsgrid3.its.uiowa.edu:2135/mds-vo-name=local,o=grid
GlueSchemaVersionMajor: 1
GlueSchemaVersionMinor: 2
\end{verbatim}
\normalsize
\caption[Example of an LDIF Representation of the GLUE
Schema]{\label{LDIF-CE-1-Example}  An LDIF respresentation
of a Computing Element entity of the GLUE schema v1.2}
\end{figure}

\begin{figure}
\scriptsize
\begin{verbatim}
# Computing Element Entity (2)
dn: GlueCEUniqueID=rsgrid3.its.uiowa.edu:2119/jobmanager-condor-fermilab,mds-vo-name=local,o=grid
objectClass: GlueCETop
objectClass: GlueCE
objectClass: GlueSchemaVersion
objectClass: GlueCEAccessControlBase
objectClass: GlueCEInfo
objectClass: GlueCEPolicy
objectClass: GlueCEState
objectClass: GlueInformationService
objectClass: GlueKey
GlueCEHostingCluster: rsgrid3.its.uiowa.edu
GlueCEName: fermilab
GlueCEUniqueID: rsgrid3.its.uiowa.edu:2119/jobmanager-condor-fermilab
GlueCEInfoGatekeeperPort: 2119
GlueCEInfoHostName: rsgrid3.its.uiowa.edu
GlueCEInfoLRMSType: condor
GlueCEInfoLRMSVersion: 6.7.13
GlueCEInfoTotalCPUs: 1
GlueCEInfoJobManager: condor
GlueCEInfoContactString: rsgrid3.its.uiowa.edu:2119/jobmanager
GlueCEInfoApplicationDir: /export/grid/app
GlueCEInfoDataDir: /export/grid/data
GlueCEInfoDefaultSE: rsgrid3.its.uiowa.edu
GlueCEStateEstimatedResponseTime: 0
GlueCEStateFreeCPUs: 0
GlueCEStateRunningJobs: 0
GlueCEStateStatus: Production
GlueCEStateTotalJobs: 0
GlueCEStateWaitingJobs: 0
GlueCEStateWorstResponseTime: 0
GlueCEStateFreeJobSlots: 1
GlueCEPolicyMaxCPUTime: 0
GlueCEPolicyMaxRunningJobs: 0
GlueCEPolicyMaxTotalJobs: 0
GlueCEPolicyMaxWallClockTime: 0
GlueCEPolicyPriority: 0
GlueCEPolicyAssignedJobSlots: 1
GlueCEAccessControlBaseRule: VO:fermilab
GlueForeignKey: GlueClusterUniqueID=rsgrid3.its.uiowa.edu
GlueInformationServiceURL: ldap://rsgrid3.its.uiowa.edu:2135/mds-vo-name=local,o=grid
GlueSchemaVersionMajor: 1
GlueSchemaVersionMinor: 2
\end{verbatim}
\normalsize
\caption[LDIF Representation of a CE Entity]{
\label{LDIF-CE-2-Example} An LDIF respresentation
of a Computing Element entity of the GLUE schema v1.2}
\end{figure}

\begin{figure}
\scriptsize
\begin{verbatim}# Cluster Entity
dn: GlueClusterUniqueID=rsgrid3.its.uiowa.edu,mds-vo-name=local,o=grid
objectClass: GlueClusterTop
objectClass: GlueCluster
objectClass: GlueSchemaVersion
objectClass: GlueInformationService
objectClass: GlueKey
GlueClusterName: rsgrid3.its.uiowa.edu
GlueClusterService: rsgrid3.its.uiowa.edu
GlueClusterUniqueID: rsgrid3.its.uiowa.edu
GlueForeignKey: GlueCEUniqueID=rsgrid3.its.uiowa.edu:2119/jobmanager-condor-cdf
GlueForeignKey: GlueCEUniqueID=rsgrid3.its.uiowa.edu:2119/jobmanager-condor-fermilab
GlueForeignKey: GlueSiteUniqueID=rsgrid3.its.uiowa.edu
GlueInformationServiceURL: ldap://rsgrid3.its.uiowa.edu:2135/mds-vo-name=local,o=grid
GlueSchemaVersionMajor: 1
GlueSchemaVersionMinor: 2

# SubCluster Entity
dn: GlueSubClusterUniqueID=rsgrid3.its.uiowa.edu, GlueClusterUniqueID=rsgrid3.its.uiowa.edu,
    mds-vo-name=local,o=grid
objectClass: GlueClusterTop
objectClass: GlueSubCluster
objectClass: GlueSchemaVersion
objectClass: GlueHostApplicationSoftware
objectClass: GlueHostArchitecture
objectClass: GlueHostBenchmark
objectClass: GlueHostMainMemory
objectClass: GlueHostNetworkAdapter
objectClass: GlueHostOperatingSystem
objectClass: GlueHostProcessor
objectClass: GlueInformationService
objectClass: GlueKey
GlueChunkKey: GlueClusterUniqueID=rsgrid3.its.uiowa.edu
GlueHostApplicationSoftwareRunTimeEnvironment: OSG-0.3.6
GlueHostApplicationSoftwareRunTimeEnvironment: ATLAS_LOC_903
GlueHostApplicationSoftwareRunTimeEnvironment: LCG-2_6_0
GlueHostArchitectureSMPSize: 2
GlueHostBenchmarkSF00: 380
GlueHostBenchmarkSI00: 400
GlueHostMainMemoryRAMSize: 512
GlueHostMainMemoryVirtualSize: 1024
GlueHostNetworkAdapterInboundIP: FALSE
GlueHostNetworkAdapterOutboundIP: TRUE
GlueHostOperatingSystemName: linux-rocks-3.3
GlueHostOperatingSystemRelease: Rocks Linux
GlueHostOperatingSystemVersion: 3.3
GlueHostProcessorClockSpeed: 1000
GlueHostProcessorModel: Pentium III (Coppermine)
GlueHostProcessorVendor: GenuineIntel
GlueSubClusterName: rsgrid3.its.uiowa.edu
GlueSubClusterUniqueID: rsgrid3.its.uiowa.edu
GlueSubClusterPhysicalCPUs: 1
GlueSubClusterLogicalCPUs: 2
GlueSubClusterTmpDir: /export/grid/data
GlueSubClusterWNTmpDir: /tmp
GlueInformationServiceURL: ldap://rsgrid3.its.uiowa.edu:2135/mds-vo-name=local,o=grid
GlueSchemaVersionMajor: 1
GlueSchemaVersionMinor: 2
\end{verbatim}
\normalsize
\caption[LDIF Representation of another CE Entity]{
\label{LDIF-Cluster-Subcluster-Example} An LDIF representation
of a Cluster and SubCluster entities of the GLUE schema v1.2}
\end{figure}

\begin{figure}
\scriptsize
\begin{verbatim}
# VOView Entity: Ignored in the mapping with the basic rules
dn: GlueVOViewLocalID=cdf,GlueCEUniqueID=rsgrid3.its.uiowa.edu:2119/
    jobmanager-condor-cdf,mds-vo-name=local,o=grid
objectClass: GlueCETop
objectClass: GlueVOView
objectClass: GlueCEInfo
objectClass: GlueCEState
objectClass: GlueCEAccessControlBase
objectClass: GlueCEPolicy
objectClass: GlueKey
objectClass: GlueSchemaVersion
GlueVOViewLocalID: cdf
GlueCEAccessControlBaseRule: VO:cdf
GlueCEStateRunningJobs: 0
GlueCEStateWaitingJobs: 0
GlueCEStateTotalJobs: 0
GlueCEStateFreeJobSlots: 1
GlueCEStateEstimatedResponseTime: 0
GlueCEStateWorstResponseTime: 0
GlueCEInfoDefaultSE: rsgrid3.its.uiowa.edu
GlueCEInfoApplicationDir: /export/grid/app
GlueCEInfoDataDir: /export/grid/data
GlueChunkKey: GlueCEUniqueID=rsgrid3.its.uiowa.edu:2119/jobmanager-condor-cdf
GlueSchemaVersionMajor: 1
GlueSchemaVersionMinor: 2

# Site Entity: Ignored in the mapping
dn: GlueSiteUniqueID=rsgrid3.its.uiowa.edu,mds-vo-name=local,o=grid
objectClass: GlueTop
objectClass: GlueSite
objectClass: GlueKey
objectClass: GlueSchemaVersion
GlueSiteUniqueID: rsgrid3.its.uiowa.edu
GlueSiteName: UIOWA-ITB
GlueSiteDescription: OSG Site
GlueSiteUserSupportContact: mailto: shaowen-wang@uiowa.edu
GlueSiteSysAdminContact: mailto: shaowen-wang@uiowa.edu
GlueSiteSecurityContact: mailto: shaowen-wang@uiowa.edu
GlueSiteLocation: Iowa City , US
GlueSiteLatitude: 41.67
GlueSiteLongitude: -91.55
GlueSiteWeb: http://rsgrid3.its.uiowa.edu
GlueForeignKey: GlueSiteUniqueID=rsgrid3.its.uiowa.edu
GlueSchemaVersionMajor: 1
GlueSchemaVersionMinor: 2
\end{verbatim}
\normalsize
\caption[Example of an LDIF Representation of the VOView and Site entities of
the  GLUE Schema]{\label{LDIF-VOView-Site-Example} An LDIF representation
of a VOView and Site entities of the GLUE schema v1.2. A mapping
for VOView and Site is not provided by the basic rules but is provided by the extended
rules.}
\end{figure}





\begin{figure}
\scriptsize
\begin{verbatim}
# Location Entity: Ignored in the mapping with the basic rules
# OSG_SITE_READ, rsgrid3.its.uiowa.edu, rsgrid3.its.uiowa.edu, local, grid
dn: GlueLocationLocalID=OSG_SITE_READ,
    GlueSubClusterUniqueID=rsgrid3.its.uiowa.edu,
    GlueClusterUniqueID=rsgrid3.its.uiowa.edu,mds-vo-name=local,o=grid
objectClass: GlueClusterTop
objectClass: GlueLocation
objectClass: GlueKey
objectClass: GlueSchemaVersion
GlueLocationLocalID: OSG_SITE_READ
GlueLocationName: OSG_SITE_READ
GlueLocationPath: /export/grid/data
GlueChunkKey: GlueClusterUniqueID=rsgrid3.its.uiowa.edu
GlueSchemaVersionMajor: 1
GlueSchemaVersionMinor: 2

# OSG_SITE_WRITE, rsgrid3.its.uiowa.edu, rsgrid3.its.uiowa.edu, local, grid
dn: GlueLocationLocalID=OSG_SITE_WRITE,
    GlueSubClusterUniqueID=rsgrid3.its.uiowa.edu,
    GlueClusterUniqueID=rsgrid3.its.uiowa.edu,mds-vo-name=local,o=grid
objectClass: GlueClusterTop
objectClass: GlueLocation
objectClass: GlueKey
objectClass: GlueSchemaVersion
GlueLocationLocalID: OSG_SITE_WRITE
GlueLocationName: OSG_SITE_WRITE
GlueLocationPath: /export/grid/data
GlueChunkKey: GlueClusterUniqueID=rsgrid3.its.uiowa.edu
GlueSchemaVersionMajor: 1
GlueSchemaVersionMinor: 2
\end{verbatim}
\normalsize \caption[Example of an LDIF Representation of the
Location entity of the GLUE Schema]{\label{LDIF-Location-Example} An
LDIF representation of Location entities of the GLUE schema v1.2.
The basic rules do not provide a mapping to old ClassAd for these
entities, but the extended rules do (see
Fig.\ref{CLASSAD-Extended-1-Example-part2}).}
\end{figure}







\begin{figure}
\scriptsize
\begin{verbatim}
### Computing Element Entity (1)
GlueCEHostingCluster = "rsgrid3.its.uiowa.edu"
GlueCEName = "cdf"
GlueCEUniqueID = "rsgrid3.its.uiowa.edu:2119/jobmanager-condor-cdf"
GlueCEInfoGatekeeperPort = 2119
GlueCEInfoHostName = "rsgrid3.its.uiowa.edu"
GlueCEInfoLRMSType = "condor"
GlueCEInfoLRMSVersion = "6.7.13"
GlueCEInfoTotalCPUs = 1
GlueCEInfoJobManager = "condor"
GlueCEInfoContactString = "rsgrid3.its.uiowa.edu:2119/jobmanager-condor"
GlueCEInfoApplicationDir = "/export/grid/app"
GlueCEInfoDataDir = "/export/grid/data"
GlueCEInfoDefaultSE = "rsgrid3.its.uiowa.edu"
GlueCEInfoGRAMVersion = "1.7"
GlueCEStateEstimatedResponseTime = 0
GlueCEStateFreeCPUs = 0
GlueCEStateRunningJobs = 0
GlueCEStateStatus = "Production"
GlueCEStateTotalJobs = 0
GlueCEStateWaitingJobs = 0
GlueCEStateWorstResponseTime = 0
GlueCEStateFreeJobSlots = 1
GlueCEPolicyMaxCPUTime = 0
GlueCEPolicyMaxRunningJobs = 0
GlueCEPolicyMaxTotalJobs = 0
GlueCEPolicyMaxWallClockTime = 0
GlueCEPolicyPriority = 0
GlueCEPolicyAssignedJobSlots = 1
GlueCEAccessControlBaseRule = "VO:cdf"
GlueInformationServiceURL = "ldap://rsgrid3.its.uiowa.edu:2135/mds-vo-name=local,o=grid"
GlueSchemaVersionMajor: 1
GlueSchemaVersionMinor: 2
### Cluster Entity
GlueClusterName = "rsgrid3.its.uiowa.edu"
GlueClusterService = "rsgrid3.its.uiowa.edu"
GlueClusterUniqueID = "rsgrid3.its.uiowa.edu"
GlueClusterTmpDir = "/tmp"
GlueClusterWNTmpDir = "/scratch"
### SubCluster Entity
GlueHostApplicationSoftwareRunTimeEnvironment = "OSG-0.3.6,ATLAS_LOC_903,LCG-2_6_0"
GlueHostArchitectureSMPSize = 2
GlueHostArchitecturePlatformType = undefined
GlueHostBenchmarkSF00 = 380
GlueHostBenchmarkSI00 = 400
GlueHostMainMemoryRAMSize = 512
GlueHostMainMemoryVirtualSize = 1024
GlueHostNetworkAdapterInboundIP = FALSE
GlueHostNetworkAdapterOutboundIP = TRUE
GlueHostOperatingSystemName = "linux-rocks-3.3"
GlueHostOperatingSystemRelease = "Rocks Linux"
GlueHostOperatingSystemVersion = 3.3
GlueHostProcessorClockSpeed = 1000
GlueHostProcessorModel = "Pentium III (Coppermine)"
GlueHostProcessorVendor = "GenuineIntel"
GlueHostProcessorVersion = undefined
GlueHostProcessorInstructionSet = "i686"
GlueHostProcessorOtherDescription = "i686"
GlueSubClusterName = "rsgrid3.its.uiowa.edu"
GlueSubClusterUniqueID = "rsgrid3.its.uiowa.edu"
GlueSubClusterPhysicalCPUs = 1
GlueSubClusterLogicalCPUs = 2
GlueSubClusterTmpDir = "/export/grid/data"
GlueSubClusterWNTmpDir = "/tmp"
\end{verbatim}
\normalsize \caption[Example of an old ClassAd representation of the
GLUE Schema]{\label{CLASSAD-Basic-1-Example} The first of two old
ClassAds representing the GLUE Schema presented before in LDIF
format. This ClassAd was generated using the ``basic rules'' for the
mapping. The 3 pound signs (\#\#\#) are used as comments only and
are not part of the ClassAd.}
\end{figure}

\begin{figure}
\scriptsize
\begin{verbatim}
### Computing Element Entity (2)
GlueCEHostingCluster = "rsgrid3.its.uiowa.edu"
GlueCEName = "fermilab"
GlueCEUniqueID = "rsgrid3.its.uiowa.edu:2119/jobmanager-condor-fermilab"
GlueCEInfoGatekeeperPort = 2119
GlueCEInfoHostName = "rsgrid3.its.uiowa.edu"
GlueCEInfoLRMSType = "condor"
GlueCEInfoLRMSVersion = "6.7.13"
GlueCEInfoTotalCPUs = 1
GlueCEInfoJobManager = "condor"
GlueCEInfoContactString = "rsgrid3.its.uiowa.edu:2119/jobmanager-condor"
GlueCEInfoApplicationDir = "/export/grid/app"
GlueCEInfoDataDir = "/export/grid/data"
GlueCEInfoDefaultSE = "rsgrid3.its.uiowa.edu"
GlueCEInfoGRAMVersion = "1.7"
GlueCEStateEstimatedResponseTime = 0
GlueCEStateFreeCPUs = 0
GlueCEStateRunningJobs = 0
GlueCEStateStatus = "Production"
GlueCEStateTotalJobs = 0
GlueCEStateWaitingJobs = 0
GlueCEStateWorstResponseTime = 0
GlueCEStateFreeJobSlots = 1
GlueCEPolicyMaxCPUTime = 0
GlueCEPolicyMaxRunningJobs = 0
GlueCEPolicyMaxTotalJobs = 0
GlueCEPolicyMaxWallClockTime = 0
GlueCEPolicyPriority = 0
GlueCEPolicyAssignedJobSlots = 1
GlueCEAccessControlBaseRule = "VO:fermilab"
GlueInformationServiceURL = "ldap://rsgrid3.its.uiowa.edu:2135/mds-vo-name=local,o=grid"
GlueSchemaVersionMajor = 1
GlueSchemaVersionMinor = 2
### Cluster Entity
GlueClusterName = "rsgrid3.its.uiowa.edu"
GlueClusterService = "rsgrid3.its.uiowa.edu"
GlueClusterUniqueID = "rsgrid3.its.uiowa.edu"
GlueClusterTmpDir = "/tmp"
GlueClusterWNTmpDir = "/scratch"
### SubCluster Entity
GlueHostApplicationSoftwareRunTimeEnvironment = "OSG-0.3.6,ATLAS_LOC_903,LCG-2_6_0"
GlueHostArchitectureSMPSize = 2
GlueHostArchitecturePlatformType = undefined
GlueHostBenchmarkSF00 = 380
GlueHostBenchmarkSI00 = 400
GlueHostMainMemoryRAMSize = 512
GlueHostMainMemoryVirtualSize = 1024
GlueHostNetworkAdapterInboundIP = FALSE
GlueHostNetworkAdapterOutboundIP = TRUE
GlueHostOperatingSystemName = "linux-rocks-3.3"
GlueHostOperatingSystemRelease = "Rocks Linux"
GlueHostOperatingSystemVersion = 3.3
GlueHostProcessorClockSpeed = 1000
GlueHostProcessorModel = "Pentium III (Coppermine)"
GlueHostProcessorVendor = "GenuineIntel"
GlueHostProcessorVersion = undefined
GlueHostProcessorInstructionSet = "i686"
GlueHostProcessorOtherDescription = "i686"
GlueSubClusterName = "rsgrid3.its.uiowa.edu"
GlueSubClusterUniqueID = "rsgrid3.its.uiowa.edu"
GlueSubClusterPhysicalCPUs = 1
GlueSubClusterLogicalCPUs = 2
GlueSubClusterTmpDir = "/export/grid/data"
GlueSubClusterWNTmpDir = "/tmp"
\end{verbatim}
\normalsize \caption[Example of an old ClassAd representation of the
GLUE Schema]{\label{CLASSAD-Basic-2-Example} The second of two old
ClassAds representing the GLUE Schema presented before in LDIF
format. This ClassAd was generated using the ``basic rules'' for the
mapping. The 3 pound signs (\#\#\#) are used as comments only and
are not part of the ClassAd.}
\end{figure}







\begin{figure}
\scriptsize
\begin{verbatim}
### Computing Element Entity (1)
GlueCEHostingCluster = "rsgrid3.its.uiowa.edu"
GlueCEName = "cdf"
GlueCEUniqueID = "rsgrid3.its.uiowa.edu:2119/jobmanager-condor-cdf"
GlueCEInfoGatekeeperPort = 2119
GlueCEInfoHostName = "rsgrid3.its.uiowa.edu"
GlueCEInfoLRMSType = "condor"
GlueCEInfoLRMSVersion = "6.7.13"
GlueCEInfoTotalCPUs = 1
GlueCEInfoJobManager = "condor"
GlueCEInfoContactString = "rsgrid3.its.uiowa.edu:2119/jobmanager-condor"
GlueCEInfoGRAMVersion = "1.7"
GlueCEStateFreeCPUs = 0
GlueCEStateStatus = "Production"
GlueCEPolicyMaxCPUTime = 0
GlueCEPolicyMaxRunningJobs = 0
GlueCEPolicyMaxTotalJobs = 0
GlueCEPolicyMaxWallClockTime = 0
GlueCEPolicyPriority = 0
GlueCEPolicyAssignedJobSlots = 1
GlueInformationServiceURL = "ldap://rsgrid3.its.uiowa.edu:2135/mds-vo-name=local,o=grid"
GlueSchemaVersionMajor: 1
GlueSchemaVersionMinor: 2
### Attributes overwritten by VOView Entity
GlueVOViewLocalID: "cdf"
GlueCEAccessControlBaseRule = "VO:cdf"
GlueCEStateRunningJobs = 0
GlueCEStateWaitingJobs = 0
GlueCEStateTotalJobs = 0
GlueCEStateFreeJobSlots = 1
GlueCEStateEstimatedResponseTime = 0
GlueCEStateWorstResponseTime = 0
GlueCEInfoDefaultSE = "rsgrid3.its.uiowa.edu"
GlueCEInfoApplicationDir = "/export/grid/app"
GlueCEInfoDataDir = "/export/grid/data"
\end{verbatim}
\normalsize \caption[Example of an old ClassAd representation of the
GLUE Schema]{\label{CLASSAD-Extended-1-Example-part1} The first half
of the first of two old ClassAds representing the GLUE Schema
presented before in LDIF format. This ClassAd was generated using
the ``extended rules'' for the mapping. The 3 pound signs (\#\#\#)
are used as comments only and are not part of the ClassAd.}
\end{figure}

\begin{figure}
\scriptsize
\begin{verbatim}
### Cluster Entity
GlueClusterName = "rsgrid3.its.uiowa.edu"
GlueClusterService = "rsgrid3.its.uiowa.edu"
GlueClusterUniqueID = "rsgrid3.its.uiowa.edu"
GlueClusterTmpDir = "/tmp"
GlueClusterWNTmpDir = "/scratch"
### SubCluster Entity
GlueHostApplicationSoftwareRunTimeEnvironment = "OSG-0.3.6,ATLAS_LOC_903,LCG-2_6_0"
GlueHostArchitectureSMPSize = 2
GlueHostArchitecturePlatformType = undefined
GlueHostBenchmarkSF00 = 380
GlueHostBenchmarkSI00 = 400
GlueHostMainMemoryRAMSize = 512
GlueHostMainMemoryVirtualSize = 1024
GlueHostNetworkAdapterInboundIP = FALSE
GlueHostNetworkAdapterOutboundIP = TRUE
GlueHostOperatingSystemName = "linux-rocks-3.3"
GlueHostOperatingSystemRelease = "Rocks Linux"
GlueHostOperatingSystemVersion = 3.3
GlueHostProcessorClockSpeed = 1000
GlueHostProcessorModel = "Pentium III (Coppermine)"
GlueHostProcessorVendor = "GenuineIntel"
GlueHostProcessorVersion = undefined
GlueHostProcessorInstructionSet = "i686"
GlueHostProcessorOtherDescription = "i686"
GlueSubClusterName = "rsgrid3.its.uiowa.edu"
GlueSubClusterUniqueID = "rsgrid3.its.uiowa.edu"
GlueSubClusterPhysicalCPUs = 1
GlueSubClusterLogicalCPUs = 2
GlueSubClusterTmpDir = "/export/grid/data"
GlueSubClusterWNTmpDir = "/tmp"
### Location Entity
GlueLocationLocalID = "OSG_SITE_READ,OSG_SITE_WRITE"
GlueLocationName = "OSG_SITE_READ,OSG_SITE_WRITE"
### Site Entity: at least these two attribute must be present
GlueSiteUniqueID: rsgrid3.its.uiowa.edu
GlueSiteName: UIOWA-ITB
\end{verbatim}
\normalsize \caption[Example of an old ClassAd representation of the
GLUE Schema]{\label{CLASSAD-Extended-1-Example-part2} The second
half of the ClassAd in figure
\ref{CLASSAD-Extended-1-Example-part1}}
\end{figure}







\begin{figure}
\scriptsize
\begin{verbatim}
### Computing Element Entity (2)
GlueCEHostingCluster = "rsgrid3.its.uiowa.edu"
GlueCEName = "fermilab"
GlueCEUniqueID = "rsgrid3.its.uiowa.edu:2119/jobmanager-condor-fermilab"
GlueCEInfoGatekeeperPort = 2119
GlueCEInfoHostName = "rsgrid3.its.uiowa.edu"
GlueCEInfoLRMSType = "condor"
GlueCEInfoLRMSVersion = "6.7.13"
GlueCEInfoTotalCPUs = 1
GlueCEInfoJobManager = "condor"
GlueCEInfoContactString = "rsgrid3.its.uiowa.edu:2119/jobmanager-condor"
GlueCEInfoApplicationDir = "/export/grid/app"
GlueCEInfoDataDir = "/export/grid/data"
GlueCEInfoDefaultSE = "rsgrid3.its.uiowa.edu"
GlueCEInfoGRAMVersion = "1.7"
GlueCEStateEstimatedResponseTime = 0
GlueCEStateFreeCPUs = 0
GlueCEStateRunningJobs = 0
GlueCEStateStatus = "Production"
GlueCEStateTotalJobs = 0
GlueCEStateWaitingJobs = 0
GlueCEStateWorstResponseTime = 0
GlueCEStateFreeJobSlots = 1
GlueCEPolicyMaxCPUTime = 0
GlueCEPolicyMaxRunningJobs = 0
GlueCEPolicyMaxTotalJobs = 0
GlueCEPolicyMaxWallClockTime = 0
GlueCEPolicyPriority = 0
GlueCEPolicyAssignedJobSlots = 1
GlueCEAccessControlBaseRule = "VO:fermilab"
GlueInformationServiceURL = "ldap://rsgrid3.its.uiowa.edu:2135/mds-vo-name=local,o=grid"
GlueSchemaVersionMajor = 1
GlueSchemaVersionMinor = 2
### VOView Entity for VO:fermilab is not defined: nothing to overwrite.
\end{verbatim}
\normalsize \caption[Example of an old ClassAd representation of the
GLUE Schema]{\label{CLASSAD-Extended-2-Example-part1} The first half
of the second of two old ClassAds representing the GLUE Schema
presented before in LDIF format. This ClassAd was generated using
the ``extended rules'' for the mapping. The 3 pound signs (\#\#\#)
are used as comments only and are not part of the ClassAd.}
\end{figure}

\begin{figure}
\scriptsize
\begin{verbatim}
### Cluster Entity
GlueClusterName = "rsgrid3.its.uiowa.edu"
GlueClusterService = "rsgrid3.its.uiowa.edu"
GlueClusterUniqueID = "rsgrid3.its.uiowa.edu"
GlueClusterTmpDir = "/tmp"
GlueClusterWNTmpDir = "/scratch"
### SubCluster Entity
GlueHostApplicationSoftwareRunTimeEnvironment = "OSG-0.3.6,ATLAS_LOC_903,LCG-2_6_0"
GlueHostArchitectureSMPSize = 2
GlueHostArchitecturePlatformType = undefined
GlueHostBenchmarkSF00 = 380
GlueHostBenchmarkSI00 = 400
GlueHostMainMemoryRAMSize = 512
GlueHostMainMemoryVirtualSize = 1024
GlueHostNetworkAdapterInboundIP = FALSE
GlueHostNetworkAdapterOutboundIP = TRUE
GlueHostOperatingSystemName = "linux-rocks-3.3"
GlueHostOperatingSystemRelease = "Rocks Linux"
GlueHostOperatingSystemVersion = 3.3
GlueHostProcessorClockSpeed = 1000
GlueHostProcessorModel = "Pentium III (Coppermine)"
GlueHostProcessorVendor = "GenuineIntel"
GlueHostProcessorVersion = undefined
GlueHostProcessorInstructionSet = "i686"
GlueHostProcessorOtherDescription = "i686"
GlueSubClusterName = "rsgrid3.its.uiowa.edu"
GlueSubClusterUniqueID = "rsgrid3.its.uiowa.edu"
GlueSubClusterPhysicalCPUs = 1
GlueSubClusterLogicalCPUs = 2
GlueSubClusterTmpDir = "/export/grid/data"
GlueSubClusterWNTmpDir = "/tmp"
### Location Entity
GlueLocationLocalID = "OSG_SITE_READ,OSG_SITE_WRITE"
GlueLocationName = "OSG_SITE_READ,OSG_SITE_WRITE"
### Site Entity: at least these two attribute must be present
GlueSiteUniqueID: rsgrid3.its.uiowa.edu
GlueSiteName: UIOWA-ITB

\end{verbatim}
\normalsize \caption[Example of an old ClassAd representation of the
GLUE Schema]{\label{CLASSAD-Extended-2-Example-part2} The second
half of the ClassAd in figure
\ref{CLASSAD-Extended-2-Example-part1}}
\end{figure}



\begin{thebibliography}{6}
\addcontentsline{toc}{section}{References}
\bibitem{GLUE-schema} http://glueschema.forge.cnaf.infn.it/
\bibitem{LDIF-mapping} http://glueschema.forge.cnaf.infn.it/Mapping/LDAP
\bibitem{XML-mapping} http://glueschema.forge.cnaf.infn.it/Mapping/XMLSchema
\bibitem{CEMon} http://grid.pd.infn.it/cemon/
\bibitem{GIP} L.~Field.
\newblock Generic Information Provider.
\newblock http://lfield.home.cern.ch/lfield/gip
\bibitem{LDIF}
G.~Good.
\newblock The LDAP Data Interchange Format (LDIF) - Technical
Specification.
\newblock IETF RFC 2849. June 2000.
\end{thebibliography}

\end{document}
